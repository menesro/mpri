%\documentclass[pdf,serpaggi,slideColor,colorBG]{prosper}
\documentclass[smaller]{beamer}
\mode<presentation>

\usetheme{Madrid}
%\beamertemplatefootpagenumber

%\usepackage{graphics,graphicx}
\usepackage{pgf}
\usepackage{array,listings}
%\usecolortheme{seahorse}
%\usecolortheme{rose}
%\usefonttheme[onlylarge]{structuresmallcapsserif}
%\usefonttheme[onlysmall]{structurebold}
%
%\setbeamerfont{title}{shape=\itshape,family=\rmfamily}
%\setbeamercolor{title}{fg=red!80!black}
%\setbeamercolor{title}{fg=red!80!black,bg=red!20!white}

%\usepackage[latin1]{inputenc}
%\usepackage{pstricks,pst-node,pst-text,pst-3d}
% pst* are for drawing within LaTeX

%\newtheorem{defin}{Definition}%[section]
%\newtheorem{theor}{Theorem}%[section]
%\newtheorem{lemma}{Lemma}%[section]
%\newtheorem{propo}{Proposition}%[section]
%\newtheorem{corol}{Corollary}%[section]
\newtheorem{remark}{Remark}%[section]
%\newtheorem{examp}{Example}%[section]
%\newenvironment{proof}
%          {\noindent{\bf Proof}\ \ \ }{$\Box${\ }}


\input macros
\input macros_logic

\newcommand{\la}{\leftarrow}
\newcommand{\ra}{\rightarrow}
\newcommand{\lora}{\longrightarrow}
\newcommand{\La}{\Leftarrow}
\newcommand{\Ra}{\Rightarrow}
\newcommand{\lla}{\longleftarrow}
\newcommand{\lra}{\leftrightarrow}
\newcommand{\Lra}{\Leftrightarrow}
%\newcommand{\hra}{\hookrightarrow}
\newcommand{\da}{\!\!\downarrow}
\newcommand{\ua}{\!\!\uparrow}
\newcommand{\ov}{\overline}
\newcommand{\un}{\underline}
\newcommand{\longra}{\longrightarrow}

\newcommand{\state}[2]{{#1\parallel #2}}

  \def\cA{{\cal A}}
  \def\cB{{\cal B}}
  \def\cC{{\cal C}}
  \def\cD{{\cal D}}
  \def\cE{{\cal E}}
  \def\cF{{\cal F}}
  \def\cI{{\cal I}}
  \def\cK{{\cal K}}
  \def\cL{{\cal L}}
  \def\cM{{\cal M}}
  \def\cN{{\cal N}}
  \def\cO{{\cal O}}
  \def\cP{{\cal P}}
  \def\cQ{{\cal Q}}
  \def\cR{{\cal R}}
  \def\cS{{\cal S}}
  \def\cT{{\cal T}}
  \def\cV{{\cal V}}

  \def\bA{{\bf A}}
  \def\bB{{\bf B}}
  \def\bC{{\bf C}}
  \def\bD{{\bf D}}
  \def\bF{{\bf F}}
  \def\bE{{\bf E}}
  \def\bG{{\bf G}}
  \def\bH{{\bf H}}
  \def\bI{{\bf I}}
  \def\bP{{\bf P}}
  \def\bL{{\bf L}}
  \def\bN{{\bf N}}
  \def\bQ{{\bf Q}}
  \def\bR{{\bf R}}
  \def\bS{{\bf S}}
  \def\bT{{\bf T}}
  \def\bX{{\bf X}}
  \def\bU{{\bf U}}
  \def\bV{{\bf V}}
  \def\bZ{{\bf Z}}
  \def\bbarX{\overline{{\bf X}}}
	\def\bbarU{\overline{{\bf U}}}
	\def\bbarR{\overline{{\bf R}}}
  \def\bsA{{\scriptstyle A}} 
  \def\bsB{{\scriptstyle B}}

  \def\sN{{\scriptstyle N}}
  \def\sZ{{\scriptstyle Z}}

  \def\ga{{\bf a}}
  \def\gb{{\bf b}}
  \def\gc{{\bf c}}
  \def\gd{{\bf d}}
  \def\ge{{\bf e}}
  \def\gf{{\bf f}}
  
  \def\ltl{\mathsf{LTL}}
	\def\ltli{\mathsf{LTL_\infty}}
	\def\ltlp{\mathsf{LTL^+}}
	\def\ltlip{\mathsf{LTL_\infty^+}}
  \def\kctlp*{\mathsf{KCTL^* P}}
	\def\akctlp*{\mathsf{\forall KCTL^* P}}
	\def\ekctlp*{\mathsf{\exists KCTL^* P}}
	\newcommand{\Paths}{\mathsf{Paths}}
	\newcommand{\Points}{\mathsf{Points}}
	
	\def\ltl{\mathsf{LTL}}
	\def\ltli{\mathsf{LTL_\infty}}
	\def\ltlp{\mathsf{LTL^+}}
	\def\ltlip{\mathsf{LTL_\infty^+}}
	
%	\definecolor{red}{rgb}{255,0,0}
	
	\newcommand{\blue}[1]{\textcolor{blue}{#1}}
	\newcommand{\red}[1]{\textcolor{red}{#1}}
	\newcommand{\green}[1]{\textcolor{teal}{#1}}

\title[SMT solvers]{SMT solvers}
\author{Constantin Enea}
\institute[IRIF]{IRIF, University Paris Diderot - Paris 7}
%					 
%					 and \\ 
%					 
%							Department of Computer Science\\
%              University ``Al.I.Cuza'' \\
%             Iasi, Romania \\
%             E-mail: \texttt{cenea@infoiasi.ro} \\
%             \vspace{3mm}
%             \hspace{-2.1cm}
%             {\normalsize Supervisors: Anatol Slissenko, University Paris 12, France }\\
%             \vspace{1mm}
%             \hspace{1.65cm}	{\normalsize Ferucio Laurentiu Tiplea, University ``Al.I.Cuza'', Romania }}
\date{ }
%\slideCaption{C. Enea. 
%      {\it Abstraction techniques for abstract data types}}

\begin{document}

\begin{frame}
	\titlepage
\end{frame}



%\section{SMT solvers}

%---------------------------------------------------------------------- SLIDE -
\begin{frame}
\frametitle{Overview of SMT solving}
\begin{itemize}
\item \red{SMT solver = SAT solver + Theory solver}
\vspace{2mm}
\begin{itemize}
\setlength\itemsep{2ex}
\item the SAT solver enumerates possible truth assignments for atomic formulas
\item the Theory solver checks whether some truth assignment is satisfiable in the
considered theory (combination of theories)
\end{itemize}
\end{itemize}
\end{frame}

%---------------------------------------------------------------------- SLIDE -
\begin{frame}
\frametitle{Naive SMT solving}
\[
x+y\geq 0\land (x=z\limp z+y=-1)\land z>3t
\]
\begin{enumerate}
\item replace atomic formulas by boolean variables:
\[
\begin{array}{ll}
x+y\geq 0\mapsto a \quad\quad& z+y=-1\mapsto c\\[2mm]
x=z\mapsto b \quad\quad& z>3t \mapsto d
\end{array}
\]
\item use the SAT solver to get a satisfying assignment of $a\land (b\limp c)\land d$:
\[
\mbox{satisfying assignment: }a,b,c,d
\]
\item convert the satisfying assignment back to arithmetic:
\[
\underbrace{x+y\geq 0\land x=z\land z+y=-1}_{\red{\mbox{not consistent}}}\land z>3t
\]
\item feed the explanation to the SAT solver, i.e., add the clause $(\neg a\land \neg b\land \neg c)$
\end{enumerate}
\end{frame}

%---------------------------------------------------------------------- SLIDE -
\begin{frame}
\frametitle{Naive SMT solving}
\[
x+y\geq 0\land (x=z\limp z+y=-1)\land z>3t
\]
\begin{enumerate}
\item replace atomic formulas by boolean variables:
\[
\begin{array}{ll}
x+y\geq 0\mapsto a \quad\quad& z+y=-1\mapsto c\\[2mm]
x=z\mapsto b \quad\quad& z>3t \mapsto d 
\end{array}
\]

\ldots
\setcounter{enumi}{3}
\item feed the explanation to the SAT solver, i.e., add the clause $(\neg a\land \neg b\land \neg c)$
\item use the SAT solver to get a satisfying assignment of 
\[
a\land (b\limp c)\land d\land (\neg a\land \neg b\land \neg c)
\]

satisfying assignment: $a,\neg b,c,d$
\item convert the satisfying assignment back to arithmetic:
\[
x+y\geq 0\land \neg (x=z)\land z+y=-1\land z>3t
\quad\red{\mbox{satisfiable}}
\]
\end{enumerate}
\end{frame}

%\frame{\tableofcontents}

\begin{frame}
\frametitle{SMT solvers}
\tableofcontents
\end{frame}

\section{DPLL SAT solvers}

%---------------------------------------------------------------------- SLIDE -
\begin{frame}
\frametitle{Checking satisfiability of boolean formulas}
\begin{itemize}
\setlength\itemsep{2mm}
\item one of the most important problems in complexity theory but also
at the basis of many practical applications (e.g., Electronic Design Automation, 
Verification, Artificial Intelligence)
\item easier to check if we consider formulas in \red{normal form}:
	\begin{itemize}
	\setlength\itemsep{1mm}
		\item Negation Normal Form (NNF)
		\item Disjunctive Normal Form (DNF)
		\item Conjunctive Normal Form (CNF)
	\end{itemize}
\end{itemize}
\end{frame}

%---------------------------------------------------------------------- SLIDE -
\begin{frame}
\frametitle{Equivalent normal forms for boolean formulas}
\onslide<1-3>{
\red{Negation Normal Form (NNF):} negations appear only in literals
\begin{itemize}
\item $\neg q\land (r\vee \neg s)$ is in NNF; $p\vee \neg(\neg r\land s)$ is not in NNF
\item conversion: $\neg\neg F \rightarrow F$, \  $\neg(F_1\land F_2)\rightarrow \neg F_1\vee\neg F_2$, \
$\neg(F_1\land F_2)\rightarrow \neg F_1\vee\neg F_2$
\end{itemize}}

\onslide<2-3>{
\medskip
\red{Disjunctive Normal Form (DNF):} disjunctions of conjunctions of literals
\begin{itemize}
\item $(p\land q)\vee (\neg r\land s)$ is in DNF; $\neg q\land (r\vee \neg s)$ is not in DNF;
\item conversion: convert to NNF and distribute $\land$ over $\vee$
\end{itemize}}

\onslide<3-3>{
\medskip
\red{Conjunctive Normal Form (CNF):}  conjunctions of disjunctions of literals
\begin{itemize}
\item $\neg q\land (r\vee \neg s)$ is in CNF; $\neg q\vee (r\land \neg s)$ is not in CNF;
\item conversion: convert to NNF and distribute $\lor$ over $\land$
\end{itemize}}
\end{frame}

%---------------------------------------------------------------------- SLIDE -
\begin{frame}
\frametitle{Checking satisfiability of DNF}
\onslide<1-3>{
\red{Claim} If a formula is in DNF, checking satisfiability is trivial}
\onslide<2-3>{
\begin{itemize}
	\item there exists a clause which does not contain a literal and its negation
\end{itemize}
}

\onslide<3-3>{
\medskip
\red{Checking satisfiability by converting to DNF is completely impractical.}
\begin{itemize}
	\item DNF conversion causes exponential blow-up in size !
	\begin{itemize}
		\item the DNF of $(F_1\vee F_2)\land (F_3\vee F_4)$ is 
		\[
		(F_1\land F_3)\vee (F_1\land F_4)\vee (F_2\land F_3)\vee (F_2\land F_4)
		\]
	\item almost as bad as truth tables !
	\end{itemize}
\end{itemize}

}
\end{frame}

%---------------------------------------------------------------------- SLIDE -
\begin{frame}
\frametitle{Checking satisfiability of CNF}

\begin{itemize}
\setlength\itemsep{3mm}
\item
Unlike DNF, it is not trivial to determine satisfiability of CNF formulas

\item
Like DNF, CNF conversion cause exponential blow-up in size
\end{itemize}

\vspace{.5cm}
\onslide<2-2>{
\hspace{.8cm}
\setbeamercolor{prog}{fg=black,bg=lightgray}
\begin{beamercolorbox}[ht=1.4cm,center,wd=10cm]{prog}
\red{Tseitin's transformation:} converts formula $F$ to \blue{equi-satisfiable}
formula $F'$ in CNF with only a \red{linear} increase in size
\vspace{3mm}
\end{beamercolorbox}

\begin{itemize}
\item[]
\begin{itemize}
\item replace a sub-formula $\psi$ by a fresh variable $p$, then add clauses
to express the constraint $p\equival \psi$
\item Example: replace $(p_1\land p_2)$ by a fresh $p$ and add the clauses
$(\neg p\vee p_1)$, $(\neg p\vee p_2)$, and $(p\vee \neg p_1\vee \neg p_2)$
\end{itemize} 
\end{itemize}

}
\end{frame}

%---------------------------------------------------------------------- SLIDE -
\begin{frame}
\frametitle{Tseitin's transformation}
\[
\begin{array}{lcl}
CNF(p,\Delta) & = & (p,\Delta) \\[1mm]
CNF(\neg \varphi,\Delta) & = & (\neg l,\Delta'),\mbox{ where }(l,\Delta')=CNF(\varphi,\Delta) \\[1mm]
CNF(\varphi_1\land\varphi_2,\Delta) & = & (p,\Delta'),\mbox{ where } \\[1mm]
&& (l_1,\Delta_1)=CNF(\varphi_1,\Delta) \\[1mm]
&& (l_2,\Delta_2)=CNF(\varphi_2,\Delta_1) \\[1mm]
&&\mbox{$p$ is fresh} \\[1mm]
&&\Delta' = \Delta_2\cup\{\neg p\vee l_1,\neg p\vee l_2,\neg l_1\vee \neg l_2\vee p\} \\[1mm]
CNF(\varphi_1\lor\varphi_2,\Delta) & = & (p,\Delta'),\mbox{ where $\ldots$} \\[1mm]
&&\Delta' = \Delta_2\cup\{\neg p\vee l_1\vee l_2,\neg l_1\vee p,\neg l_2\vee p\} \\[1mm]
\end{array}
\]
\begin{theorem}
If $CNF(\varphi,\emptyset)=(l,\Delta)$ then $\varphi$ and $l\land \Delta$ are equi-satisfiable.
\end{theorem}
\end{frame}

%---------------------------------------------------------------------- SLIDE -
\begin{frame}
\frametitle{Tseitin's transformation - example}
$CNF(\overbrace{{\color<2-2>{red}{\underbrace{(q_1\land r_1)}_{p_2}}}\vee {\color<1-1>{red}{\underbrace{(q_2\land r_2)}_{p_1}}}}^{p_3},\emptyset)=$

\[
\begin{array}{ll}
(p_3, & \{ \onslide<1-3>{\color<1-1>{red}{\neg p_1\vee q_2,}\\[1mm]
&{\color<1-1>{red}{ \neg p_1\vee r_2,}}\\[1mm]
& {\color<1-1>{red}{ \neg q_2 \vee \neg r_2 \vee p_1,}}}\\[1mm]
\onslide<2-3>{& \color<2-2>{red}{ \neg p_2\vee q_1,}\\[1mm]
& {\color<2-2>{red}{\neg p_2\vee r_1,}}\\[1mm]
& {\color<2-2>{red}{ \neg q_1 \vee \neg r_1 \vee p_2,}}}\\[1mm]
\onslide<3-3>{& \color<3-3>{red}{ \neg p_3 \vee p_1 \vee p_2,}\\[1mm]
&{\color<3-3>{red}{ \neg p_1\vee p_3,}}\\[1mm]
&{\color<3-3>{red}{ \neg p_2\vee p_3}}}\})\\[1mm]
\end{array}
\]
\end{frame}

%---------------------------------------------------------------------- SLIDE -
\begin{frame}
\frametitle{DPLL [Davis and Putnam 1960; Davis et al 1962]}
\begin{itemize}
\item most state-of-the-art SAT solvers are based on variations of DPLL 
\end{itemize}

\vspace{2mm}
\red{Search}

\vspace{2mm}
Find a satisfying assignment by searching through all possible assignments

\vspace{2mm}
\red{Deduction}

\vspace{2mm}
Apply proof rules to deduce new facts from a set of known facts

\vspace{6mm}
\red{DPLL} combines search and deduction in an effective way !

\end{frame}

%---------------------------------------------------------------------- SLIDE -
\begin{frame}
\frametitle{Deduction in DPLL}
\begin{itemize}
\setlength\itemsep{2mm}
\item Deductive principle underlying DPLL is \red{unit resolution}
\vspace{4mm}
\begin{itemize}
	\setlength\itemsep{2mm}
	\item $\red{C_1}: p$\hspace{2cm} $\red{C_2}: (l_1\vee\ldots \neg p\ldots \vee l_n)$
	\item resolvent: $(l_1\vee\ldots\vee l_n)$
\end{itemize}
\item If we perform unit resolution using $C_1$ and $C_2$, we delete them and replace with
resolvent
\item \red{Boolean Constraint Propagation (BCP)} = all possible applications of unit resolution
\end{itemize}

\begin{example}
\vspace{1mm}
\begin{center}
$
(p)\land (\neg p\vee q)\land (r\vee \neg q\vee s)
$
\end{center}
\begin{itemize}
\item the resolvent of the first two clauses is $q$
\item applying BCP, we obtain $q\land (r\vee \neg q\vee s)$
\item the new resolvent is $(r\vee s)$.
\end{itemize}
\end{example}
\end{frame}


%---------------------------------------------------------------------- SLIDE -
\begin{frame}
\frametitle{Pure literal propagation}
\red{Pure literal propagation (PLP):}
\vspace{4mm}
\begin{itemize}
\setlength\itemsep{4mm}
\item If variable $p$ occurs only positively (i.e., the formula contains no $\neg p$) then
$p$ must be set to $\top$
\item If variable $p$ occurs only negatively (i.e., the formula contains no $p$) then
$p$ must be set to $\bot$
\end{itemize}
\end{frame}

%---------------------------------------------------------------------- SLIDE -
\begin{frame}[fragile]
\frametitle{Basic DPLL}
%    \lstset{numbers=left, 
%            numberstyle=\tiny\tt,  %\tiny
%            stepnumber=1, 
%            firstnumber=1,
%            numberfirstline=true,
%            numbersep=4pt}
\begin{lstlisting}[mathescape]
bool DPLL($\varphi$) 
{
    1. $\varphi'$=BCP($\varphi$);
    2. $\varphi''$=PLP($\varphi'$);
    3. if ($\varphi''$ = $\top$) then return SAT;
    4. else if ($\varphi''$ = $\bot$) then return UNSAT;
    5. p = choose_var($\varphi''$);
    6. if (DPLL($\varphi''$[$p\mapsto \top$]) then return SAT;
    7. else return DPLL($\varphi''$[$p\mapsto \bot$]
}
\end{lstlisting}
\end{frame}

%%---------------------------------------------------------------------- SLIDE -
%\begin{frame}
%\frametitle{Basic DPLL - example}
%\begin{itemize}
%\item 
%\end{itemize}
%\end{frame}

\section{Abstract DPLL}

\begin{frame}
\frametitle{SMT solvers}
\tableofcontents[currentsection]
\end{frame}



%---------------------------------------------------------------------- SLIDE -
\begin{frame}
\frametitle{Abstract DPLL}
\begin{itemize}
\setlength\itemsep{4mm}
\item During search, states of the DPLL procedure are of the form $\Box$ or $\state{M}{F}$
\begin{itemize}
	\item $M$ is a sequence of literals denoting a partial assignment (does not contain a literal and its negation)
	\item $F$ is a CNF formula
\end{itemize}
\item Literals in $M$ are marked if they are decision literals or not ($l^d$ is a decision literal)
\item A clause $C$ is conflicting in a state $\state{M}{F}$ if $M\models \neg C$ 
\item the DPLL procedure is modeled as a transition system over states; $\Longrightarrow$ denotes the transition relation
\end{itemize}
\end{frame}

%---------------------------------------------------------------------- SLIDE -
\begin{frame}
\frametitle{Abstract DPLL}
\small
\[
\begin{array}{lclcl}
\red{\mbox{\tt Unit Propagate:}} \\
\state{M}{F,C\vee l}&\Longrightarrow& \state{M\,l}{F,C\vee l} &\mbox{if} &\left\{\begin{array}{l}
														      M\models \neg C\\
														      \mbox{$l$ is undefined in $M$}																	      \end{array}\right. \\[2mm]
\red{\mbox{\tt PLP:}} \\
\state{M}{F}&\Longrightarrow& \state{M\,l}{F} &\mbox{if} &\left\{\begin{array}{l}
												\mbox{$l$ occurs in some clause of $F$}\\
												\mbox{$\neg l$ occurs in no clause of $F$} \\
												\mbox{$l$ is undefined in $M$}
												\end{array}\right. \\[2mm]
\red{\mbox{\tt Decide:}} \\
\state{M}{F}&\Longrightarrow& \state{M\,l^d}{F} &\mbox{if} &\left\{\begin{array}{l}
												\mbox{$l$ or $\neg l$ occurs in a clause of $F$}\\
												\mbox{$l$ is undefined in $M$}
												\end{array}\right. \\[2mm]
\red{\mbox{\tt Fail:}} \\
\state{M}{F,C}&\Longrightarrow& \Box &\mbox{if} &\left\{\begin{array}{l}
											M\models \neg C\\
											\mbox{$M$ contains no decision literals}															        \end{array}\right. \\[2mm]
\red{\mbox{\tt Backtrack:}} \\
\state{M\, l^d\, N}{F,C}&\Longrightarrow& \state{M\,\neg l}{F,C} &\mbox{if} &\left\{\begin{array}{l}
														      M\, l^d\, N\models \neg C\\
														      \mbox{$N$ contains no decision literals}															     \end{array}\right. 
\end{array}
\]
Check satisfiability: generate derivation $\state{\emptyset}{F}\Longrightarrow\ldots\Longrightarrow S_f$, where $S_f$ is final w.r.t $\Longrightarrow$
\end{frame}

%---------------------------------------------------------------------- SLIDE -
\begin{frame}
\frametitle{Abstract DPLL - example}
\[
\begin{array}{lclcl}
\onslide<1-9>{
\emptyset &\parallel & \overline{1}\vee \overline{2},\ 2\vee 3,\ \overline{1}\vee \overline{3}\vee 4,\ 2\vee \overline{3}\vee \overline{4},\ 1\vee 4 &\Longrightarrow& (\mbox{\tt Decide}) \\[2mm]}
\onslide<2-9>{
1^d &\parallel & \red{\overline{1}\vee \overline{2}},\ 2\vee 3,\ \overline{1}\vee \overline{3}\vee 4,\ 2\vee \overline{3}\vee \overline{4},\ 1\vee 4 &\Longrightarrow& (\mbox{\tt Unit Propagate}) \\[2mm]}
\onslide<3-9>{
1^d\, \overline{2} &\parallel & \overline{1}\vee \overline{2},\ \red{2\vee 3},\ \overline{1}\vee \overline{3}\vee 4,\ 2\vee \overline{3}\vee \overline{4},\ 1\vee 4 &\Longrightarrow& (\mbox{\tt Unit Propagate}) \\[2mm]}
\onslide<4-9>{
1^d\, \overline{2}\, 3 &\parallel & \overline{1}\vee \overline{2},\ 2\vee 3,\ \red{\overline{1}\vee \overline{3}\vee 4},\ 2\vee \overline{3}\vee \overline{4},\ 1\vee 4 &\Longrightarrow& (\mbox{\tt Unit Propagate}) \\[2mm]}
\onslide<5-9>{
1^d\, \overline{2}\, 3\, 4 &\parallel & \overline{1}\vee \overline{2},\ 2\vee 3,\ \overline{1}\vee \overline{3}\vee 4,\ \red{2\vee \overline{3}\vee \overline{4}},\ 1\vee 4 &\Longrightarrow& (\mbox{\tt Backtrack}) \\[2mm]}
\onslide<6-9>{
\overline{1} &\parallel & \overline{1}\vee \overline{2},\ 2\vee 3,\ \overline{1}\vee \overline{3}\vee 4,\ 2\vee \overline{3}\vee \overline{4},\ \red{1\vee 4} &\Longrightarrow& (\mbox{\tt Unit Propagate}) \\[2mm]}
\onslide<7-9>{
\overline{1}\, 4 &\parallel & \overline{1}\vee \overline{2},\ 2\vee 3,\ \overline{1}\vee \overline{3}\vee 4,\ 2\vee \overline{3}\vee \overline{4},\ 1\vee 4 &\Longrightarrow& (\mbox{\tt Decide}) \\[2mm]}
\onslide<8-9>{
\overline{1}\, 4\, \overline{3}^d &\parallel & \overline{1}\vee \overline{2},\ \red{2\vee 3},\ \overline{1}\vee \overline{3}\vee 4,\ 2\vee \overline{3}\vee \overline{4},\ 1\vee 4 &\Longrightarrow& (\mbox{\tt Unit Propagate}) \\[2mm]}
\onslide<9-9>{
\overline{1}\, 4\, \overline{3}^d\, 2 &\parallel & \overline{1}\vee \overline{2},\ 2\vee 3,\ \overline{1}\vee \overline{3}\vee 4,\ 2\vee \overline{3}\vee \overline{4},\ 1\vee 4 & & \red{\mbox{Satisfiable}} \\[2mm]}

\end{array}
\]
\end{frame}

%---------------------------------------------------------------------- SLIDE -
\begin{frame}
\frametitle{Abstract DPLL - adding backjumping}
\onslide<1-9>{
\begin{itemize}
\item Usually the PLP is used only as a pre-processing step
\item Chronological backtracking (rule {\tt Backtrack}) is replaced by \red{backjumping}
\end{itemize}}
\[
\begin{array}{lclcl}
\onslide<2-9>{
\emptyset &\parallel & \overline{1}\vee 2,\ \overline{3}\vee 4,\ \overline{5}\vee \overline{6},\ 6\vee \overline{5}\vee \overline{2}, &\Longrightarrow& (\mbox{\tt Decide}) \\[2mm]}
\onslide<3-9>{
1^d &\parallel & \red{\overline{1}\vee 2},\ \overline{3}\vee 4,\ \overline{5}\vee \overline{6},\ 6\vee \overline{5}\vee \overline{2}, &\Longrightarrow& (\mbox{\tt Unit Propagate}) \\[2mm]}
\onslide<4-9>{
1^d\, 2 &\parallel & \overline{1}\vee 2,\ \overline{3}\vee 4,\ \overline{5}\vee \overline{6},\ 6\vee \overline{5}\vee \overline{2}, &\Longrightarrow& (\mbox{\tt Decide}) \\[2mm]}
\onslide<5-9>{
1^d\, 2\, 3^d &\parallel & \overline{1}\vee 2,\ \red{\overline{3}\vee 4},\ \overline{5}\vee \overline{6},\ 6\vee \overline{5}\vee \overline{2}, &\Longrightarrow& (\mbox{\tt Unit Propagate}) \\[2mm]}
\onslide<6-9>{
1^d\, 2\, 3^d\, 4 &\parallel & \overline{1}\vee 2,\ \overline{3}\vee 4,\ \overline{5}\vee \overline{6},\ 6\vee \overline{5}\vee \overline{2}, &\Longrightarrow& (\mbox{\tt Decide}) \\[2mm]}
\onslide<7-9>{
1^d\, 2\, 3^d\, 4\, 5^d &\parallel & \overline{1}\vee 2,\ \overline{3}\vee 4,\ \red{\overline{5}\vee \overline{6}},\ 6\vee \overline{5}\vee \overline{2}, &\Longrightarrow& (\mbox{\tt Unit Propagate}) \\[2mm]}
\onslide<8-9>{
1^d\, 2\, 3^d\, 4\, 5^d \overline{6} &\parallel & \overline{1}\vee 2,\ \overline{3}\vee 4,\ \overline{5}\vee \overline{6},\ \red{6\vee \overline{5}\vee \overline{2}}, &\Longrightarrow& (\mbox{\tt Backtrack}) \\[2mm]}
\onslide<9-9>{
1^d\, 2\, 3^d\, 4\, \overline{5} &\parallel & \overline{1}\vee 2,\ \overline{3}\vee 4,\ \overline{5}\vee \overline{6},\ 6\vee \overline{5}\vee \overline{2}, & &  \\[2mm]}
\end{array}
\]
\end{frame}

%---------------------------------------------------------------------- SLIDE -
\begin{frame}
\frametitle{Abstract DPLL - adding backjumping}
\onslide<1-2>{
\[
{\footnotesize
\begin{array}{lclcl}
\emptyset &\parallel & \overline{1}\vee 2,\ \overline{3}\vee 4,\ \overline{5}\vee \overline{6},\ 6\vee \overline{5}\vee \overline{2}, &\Longrightarrow& (\mbox{\tt Decide}) \\[2mm]
1^d &\parallel & \red{\overline{1}\vee 2},\ \overline{3}\vee 4,\ \overline{5}\vee \overline{6},\ 6\vee \overline{5}\vee \overline{2}, &\Longrightarrow& \fbox{(\mbox{\tt Unit Propagate})} \\[2mm]
1^d\, 2 &\parallel & \overline{1}\vee 2,\ \overline{3}\vee 4,\ \overline{5}\vee \overline{6},\ 6\vee \overline{5}\vee \overline{2}, &\Longrightarrow& (\mbox{\tt Decide}) \\[2mm]
1^d\, 2\, 3^d &\parallel & \overline{1}\vee 2,\ \red{\overline{3}\vee 4},\ \overline{5}\vee \overline{6},\ 6\vee \overline{5}\vee \overline{2}, &\Longrightarrow& (\mbox{\tt Unit Propagate}) \\[2mm]
1^d\, 2\, 3^d\, 4 &\parallel & \overline{1}\vee 2,\ \overline{3}\vee 4,\ \overline{5}\vee \overline{6},\ 6\vee \overline{5}\vee \overline{2}, &\Longrightarrow& \fbox{(\mbox{\tt Decide})} \\[2mm]
1^d\, 2\, 3^d\, 4\, 5^d &\parallel & \overline{1}\vee 2,\ \overline{3}\vee 4,\ \red{\overline{5}\vee \overline{6}},\ 6\vee \overline{5}\vee \overline{2}, &\Longrightarrow& \fbox{(\mbox{\tt Unit Propagate})} \\[2mm]
1^d\, 2\, 3^d\, 4\, 5^d \overline{6} &\parallel & \overline{1}\vee 2,\ \overline{3}\vee 4,\ \overline{5}\vee \overline{6},\ \red{6\vee \overline{5}\vee \overline{2}}, &\Longrightarrow& (\mbox{\tt Backtrack}) \\[2mm]
1^d\, 2\, 3^d\, 4\, \overline{5} &\parallel & \overline{1}\vee 2,\ \overline{3}\vee 4,\ \overline{5}\vee \overline{6},\ 6\vee \overline{5}\vee \overline{2}, & &  \\
\end{array}}
\]}
\onslide<2-2>{
\begin{itemize}
\item Infer that decision $1^d$ is incompatible with the decision $5^d$, i.e., the formula implies $\overline{1}\vee\overline{5}$. Similarly, it entails $\overline{2}\vee\overline{5}$.

\item \red{Backjump clauses}: their presence would have allowed unit propagation at an earlier decision level
\item Ex: given $\overline{2}\vee\overline{5}$, backjump to $1^d\, 2\, \overline{5}$
\end{itemize}}
\end{frame}

%---------------------------------------------------------------------- SLIDE -
\begin{frame}
\frametitle{Backjump rule}
\small
\[
\begin{array}{lclcl}
\red{\mbox{\tt Backjump:}} \\
\state{M\, l^d\, N}{F,C}&\Longrightarrow& \state{M\,l'}{F,C} &\mbox{if} &\left\{\begin{array}{l}
												      M\, l^d\, N\models \neg C,\\
												      \mbox{there is some clause $C'\vee l'$ s.t.} \\
												\ \ \ \mbox{$F,C\models C'\vee l'$ and $M\models \neg C'$,}\\    	 											   \ \ \ \mbox{$l'$ is undefined in $M$}\\   				
											\ \ \  \mbox{$l'$ or $\neg l'$ occurs in $F$ or in $M\, l^d\, N$}								      				           \end{array}\right. \\[2mm]
\end{array}
\]
\begin{itemize}
\item it goes back more than one decision level
\item it replaces {\tt Backtrack}
\item ``implementation problem'': generate small backjump clauses that ``jump'' many decision levels
\end{itemize}
\end{frame}

%---------------------------------------------------------------------- SLIDE -
\begin{frame}
\frametitle{Learning backjump clauses}
\begin{itemize}
\item most modern DPLL implementations \red{learn} backjump clauses (i.e., add them to the set of learned clauses)
\begin{itemize}
	\item prevent further similar conflicts
	\item when such conflicts are not likely to be found again, the clauses can be removed.
\end{itemize}
\end{itemize}
\[
\begin{array}{lclcl}
\red{\mbox{\tt Learn:}} \\
\state{M}{F}&\Longrightarrow& \state{M}{F,C} &\mbox{if} &\left\{\begin{array}{l}
												F\models C\\
												\mbox{each atom of $C$ occurs in $F$ or $M$}															\end{array}\right. \\[2mm]
\red{\mbox{\tt Forget:}} \\
\state{M}{F,C}&\Longrightarrow& \state{M}{F} &\mbox{if} &\left\{\begin{array}{l}
												F\models C\\
												\end{array}\right. \\[2mm]
\end{array}
\]
\begin{itemize}
	\item {\tt Learn} is a more general rule that allows other techniques producing consequences of $F$
	\item their application is limited in practice by the cost of proving the entailments
\end{itemize}
\end{frame}

%---------------------------------------------------------------------- SLIDE -
\begin{frame}
\frametitle{Proving termination}
\begin{theorem}
Every derivation of DPLL is finite if it contains no infinite sub-derivations consisting of only {\tt Learn} and {\tt Forget}.
\end{theorem}
%\begin{itemize}
%\item Define a \red{well-founded} partial order relation $\prec$ on states $\state{M}{F}$ ($\Box$ is a minimal element)
%\item given an assignment $N$, $m(N)=nb\_literals(F)-length(N)$ (number of literals ``missing'' in $N$ for $N$ to be total
%\item Given
%\[
%\begin{array}{l}
%M=M_0 l_1 M_1\ldots l_p M_p,\mbox{ where $l_1$,$\ldots$,$l_p$ are all the decision literals of $M$}\\[2mm]
%M'=M_0' l_1' M_1'\ldots l_{p'}' M_{p'},\mbox{ where $l_1'$,$\ldots$,$l_{p'}'$ are all the decision literals of $M'$}\\[2mm]
%\end{array}
%\]
%we have that \red{$\state{M'}{F'} \prec \state{M}{F}$} if
%\[
%\begin{array}{l}
%\mbox{(1) there is some $0\leq i\leq p,p'$ such that} \\[2mm]
%\quad\quad m(M_0)=m(M_0'),\ldots m(M_{i-1})=m(M'_{i-1}), m(M_i') < m(M_i)\mbox{ or}\\[2mm]
%\mbox{(2) }p<p', m(M_0)=m(M_0'),\ldots m(M_p)=m(M'_p)\mbox{ and }m(M')<m(M) 
%\end{array}
%\] 
%\end{itemize}
\end{frame}

%---------------------------------------------------------------------- SLIDE -
\begin{frame}
\frametitle{Practical implementations and restarts}
\begin{itemize}
\item \red{heuristics} for choosing the decision literal when applying {\tt Decide}
\item \red{restart} the procedure when the search is not making enough progress according to some measure
\begin{itemize}
	\item upon each restart, the additional learned clauses will lead the heuristics for {\tt Decide}
	to behave differently
	\item without restart, any derivation leading to $\Box$ is equivalent to a tree-like proof by resolution
	\item with restart, derivations leading to $\Box$ are equivalent to DAG-like proofs, which can be \blue{exponentially
	more succinct}
\end{itemize}
\end{itemize}

\hspace{6mm}
$
\begin{array}{lcl}
\red{\mbox{\tt Restart:}} \\
\state{M}{F}&\Longrightarrow& \state{\emptyset}{F} 
\end{array}
$
\end{frame}

%---------------------------------------------------------------------- SLIDE -
\begin{frame}
\frametitle{Proving termination}
\begin{theorem}
Every derivation of DPLL with {\tt Restart} is finite if it contains no infinite sub-derivations consisting of only {\tt Learn} and {\tt Forget}, and {\tt Restart} has \red{increasing periodicity} in it.
\end{theorem}
%\begin{itemize}
%\item by contradiction, assume that there exists an infinite derivation
%\item without {\tt Restart}, the first component of a state $\state{M}{F}$ either decreases w.r.t. $\prec$ or remains constant
%\item no infinite sub-derivations of {\tt Learn} and {\tt Forget} $\limp$ an infinite number of {\tt Restart}
%\item after {\tt Learn} and {\tt Forget} ``have finished'', the sub-derivations between two applications of {\tt Restart} have length at most $nb\_literals(F)$ $\leadsto$ contradicts the increasing periodicity of {\tt Restart}
%\end{itemize}
\end{frame}

\section{DPLL + Theory solvers}

\begin{frame}
\frametitle{SMT solvers}
\tableofcontents[currentsection]
\end{frame}



%---------------------------------------------------------------------- SLIDE -
\begin{frame}
\frametitle{Preliminaries - signature}
\begin{itemize}
\setlength\itemsep{2mm}
\item A \red{signature} $\Sigma$ is a finite set of:
\vspace{2mm}
\begin{itemize}
\setlength\itemsep{2mm}
	\item Function symbols: $\Sigma_F=\{f,g,\ldots\}$
	\item Predicate symbols: $\Sigma_P=\{p,q,\ldots\}$ ($=$ is a logical symbol not in $\Sigma_P$)
	\item and an arity function $\pi:\Sigma\flc\Nat$
\end{itemize}
\item Function symbols of arity 0 are called \red{constants}
\item $\cV$ denotes the set of \red{variables} (disjoint from $\Sigma$)
\end{itemize}
\end{frame}

%---------------------------------------------------------------------- SLIDE -
\begin{frame}
\frametitle{Preliminaries - terms, atomic formulas}
\begin{itemize}
\onslide<1-2>{
\setlength\itemsep{2mm}
\item The set $T(\Sigma,\cV)$ of \red{terms} is the smallest set such that:
\vspace{2mm}
\begin{itemize}
	\setlength\itemsep{2mm}
	\item $\cV\subseteq T(\Sigma,\cV)$
	\item if $f\in\Sigma_F$, $t_1,\ldots,t_n\in T(\Sigma,\cV)$, and $\pi(f)=n$ then $f(t_1,\ldots,t_n)\in T(\Sigma,\cV)$
\end{itemize}
\item The set of \red{ground terms} is defined as $T(\Sigma,\emptyset)$ }
\onslide<2-2>{
\item The set of \red{atomic formulas}:
\vspace{2mm}
	\begin{itemize}
	\setlength\itemsep{2mm}
		\item $true$ and $\mathit{false}$ are atomic formulas
		\item if $p\in\Sigma_P$, $t_1,\ldots,t_n\in T(\Sigma,\cV)$, and $\pi(p)=n$ then $p(t_1,\ldots,t_n)$ is an atomic formula
	\end{itemize}
\item If $t_1,\ldots,t_n$ are ground terms then $p(t_1,\ldots,t_n)$ is called a \red{ground formula}
\item A \red{literal} is an atomic formula or its negation}
\end{itemize}
\end{frame}

%---------------------------------------------------------------------- SLIDE -
\begin{frame}
\frametitle{Preliminaries - formulas}
\begin{itemize}
	\setlength\itemsep{2mm}
\item The set of \red{formulas} over $\cF(\Sigma,\cV)$ is the smallest set such that:
\vspace{2mm}
\begin{itemize}
	\setlength\itemsep{2mm}
	\item every atomic formula is in $\cF(\Sigma,\cV)$
	\item if $\varphi_1,\varphi_2\in \cF(\Sigma,\cV)$ then $\neg\varphi_1$, $\varphi_1\land\varphi_2$, $\exists x\sep\varphi_1$, and $\forall x\sep\varphi_1$ are in $\cF(\Sigma,\cV)$
\end{itemize}
\item A formula without quantifiers is called \red{quantifier-free}
\item A variable is \red{free} iff it is not bound by a quantifier
\item A \red{sentence} is a formula with no free variables
\end{itemize}
\end{frame}

%---------------------------------------------------------------------- SLIDE -
\begin{frame}
\frametitle{Preliminaries - models}
\begin{itemize}
	\setlength\itemsep{2mm}
\item A \red{model} is defined as:
\vspace{2mm}
	\begin{itemize}
	\setlength\itemsep{2mm}
		\item $|M|$ is a set of elements
		\item function interpretations: $M(f):|M|^n\flc |M|$, for each $f\in\Sigma_F$ with $\pi(f)=n$
		\item predicate interpretations: $M(p)\subseteq |M|^n$, for each $p\in\Sigma_P$ with $\pi(p)=n$
		\item variable interpretations: $M(x)\in |M|$, for each $x\in\cV$
	\end{itemize}
\item A formula $\varphi$ is \red{true in a model $M$} if it evaluates to $true$ under $M$'s interpretations of signature symbols and variables
\item A formula $\varphi$ is \red{satisfiable} iff there exists a model $M$ such that $M\models \varphi$
\end{itemize}
\end{frame}

%---------------------------------------------------------------------- SLIDE -
\begin{frame}
\frametitle{Preliminaries - theories}
\begin{itemize}
\setlength\itemsep{2mm}
\item A \red{(first-order) theory} $\cT$ over a signature $\Sigma$ is a set of (deductively closed)
sentences over $\Sigma$ and $\cV$
\item A theory $\cT$ is \red{consistent} iff $\mathit{false}\not\in \cT$
\item Due to the (semantical) completeness of first-order logic we can view $\cT$ as the class of all models of $\cT$
\item given $M$ a model of $\cT$, \red{$M\models_\cT \varphi$} iff $M$ satisfies $\varphi$
\item given $\varphi_1$, $\varphi_2$ formulas, \red{$\varphi_1\models_\cT\varphi_2$} iff any model of $\cT$ that satisfies $\varphi_1$ satisfies also $\varphi_2$
\end{itemize}
\end{frame}

%---------------------------------------------------------------------- SLIDE -
\begin{frame}
\frametitle{Satisfiability modulo theories}
\begin{itemize}
\setlength\itemsep{2mm}
\item A formula $\varphi(\vec{x})$ is \red{satisfiable in a theory $\cT$} iff $\cT\land \exists\vec{x}\sep\varphi(\vec{x})$ is satisfiable
\vspace{2mm}
	\begin{itemize}
	\setlength\itemsep{2mm}
		\item i.e., there exists a model $M$ of $\cT$ in which $\varphi(\vec{x})$ evaluates to $true$, 
		\[
			M\models_\cT \varphi(\vec{x})
		\]
	\end{itemize}
\item \red{Satisfiability modulo $\cT$ (or $\cT$-satisfiability)} = the problem of checking satisfiability in the theory $\cT$
\item in the following, we consider problem of satisfiability modulo $\cT$ for \red{quantifier-free formulas}
\item extending Abstract DPLL over states of the form $\state{M}{F}$, where $M$ is a sequence of atomic formulas (instead of propositional variables)
\end{itemize}
\end{frame}

%---------------------------------------------------------------------- SLIDE -
\begin{frame}
\frametitle{Abstract DPLL modulo theories}
{\scriptsize
\[
\begin{array}{lclcl}
\red{\mbox{\tt Unit Propagate:}} \\
\state{M}{F,C\vee l}&\Longrightarrow& \state{M\,l}{F,C\vee l} &\mbox{if} &\left\{\begin{array}{l}
														      M\models \neg C\\
														      \mbox{$l$ is undefined in $M$}																	      \end{array}\right. \\[1mm]
\red{\mbox{\tt Decide:}} \\
\state{M}{F}&\Longrightarrow& \state{M\,l^d}{F} &\mbox{if} &\left\{\begin{array}{l}
												\mbox{$l$ or $\neg l$ occurs in a clause of $F$}\\
												\mbox{$l$ is undefined in $M$}
												\end{array}\right. \\[1mm]
\red{\mbox{\tt Fail:}} \\
\state{M}{F,C}&\Longrightarrow& \Box &\mbox{if} &\left\{\begin{array}{l}
											M\models \neg C\\
											\mbox{$M$ contains no decision literals}															        \end{array}\right. \\[1mm]
\red{\mbox{\tt Restart:}} \\
\state{M}{F}&\Longrightarrow& \state{\emptyset}{F}  
\end{array}
\]}
{\small
\[
\begin{array}{lclcl}
\red{\mbox{\tt T-Learn:}} \\
\state{M}{F}&\Longrightarrow& \state{M}{F,C} &\mbox{if} &\left\{\begin{array}{l}
												F\blue{\models_\cT}\ C\\
												\mbox{each atom of $C$ occurs in $F$ or $M$}															\end{array}\right. \\[2mm]
\red{\mbox{\tt T-Forget:}} \\
\state{M}{F,C}&\Longrightarrow& \state{M}{F} &\mbox{if} &\left\{\begin{array}{l}
												F\blue{\models_\cT}\ C\\
												\end{array}\right. \\[2mm]
\red{\mbox{\tt T-Backjump:}} \\
\state{M\, l^d\, N}{F,C}&\Longrightarrow& \state{M\,l'}{F,C} &\mbox{if} &\left\{\begin{array}{l}
												      M\, l^d\, N\models \neg C,\\
												      \mbox{there is some clause $C'\vee l'$ s.t.} \\
												\ \ \ \mbox{$F,C\blue{\models_\cT}\ C'\vee l'$ and $M\models \neg C'$,}\\    	 											   \ \ \ \mbox{$l'$ is undefined in $M$}\\   				
											\ \ \  \mbox{$l'$ or $\neg l'$ occurs in $F$ or in $M\, l^d\, N$}								      				           \end{array}\right. 
\end{array}
\]}
\end{frame}

%---------------------------------------------------------------------- SLIDE -
\begin{frame}
\frametitle{Check satisfiability}
\begin{itemize}
\setlength\itemsep{2mm}
\item let $\state{M}{F}$ be a state \red{final} w.r.t. {\tt Decide}, {\tt Fail}, {\tt Unit Propagate}, and {\tt T-Backjump}
\vspace{2mm}
	\begin{itemize}
	\setlength\itemsep{2mm}
		\item if $M$ is satisfiable in $\cT$ then $F$ is satisfiable in $\cT$
		\item otherwise, there exists a subset $l_1$,$\ldots$,$l_n$ of $M$ s.t. $\emptyset \models_\cT \neg l_1\vee \ldots\vee \neg l_n$
			\begin{itemize}
				\item apply {\tt T-Learn} and {\tt Restart}
			\end{itemize}
	\end{itemize}
\item the same termination arguments hold
\end{itemize}
\end{frame}

%---------------------------------------------------------------------- SLIDE -
\begin{frame}
\frametitle{Improvements}
\begin{itemize}
\setlength\itemsep{2mm}
\item Incremental $\cT$-solvers: check $\cT$-inconsistency even in non-final states
\vspace{2mm}
	\begin{itemize}
	\setlength\itemsep{2mm}
	\item if $M\models_\cT F$ is a state such that $M$ is not satisfiable in $\cT$, apply {\tt T-Learn} and {\tt Restart}
	\end{itemize}
\item Online SAT solvers: when $\cT$-inconsistency is detected, apply {\tt Backjump}
\item Theory propagation: add literals implied by the formula
\[
\begin{array}{lclcl}
\red{\mbox{\tt Theory Propagate:}} \\
\state{M}{F}&\Longrightarrow& \state{M\,l}{F} &\mbox{if} &\left\{\begin{array}{l}
												      M\models_\cT l,\\
												      \mbox{$l$ or $\neg l$ occurs in $F$} \\
												      \mbox{$l$ is undefined in $M$} 							      								              \end{array}\right. 
\end{array}
\]
\end{itemize}
\end{frame}

%---------------------------------------------------------------------- SLIDE -
\begin{frame}
\frametitle{Abstract DPLL modulo theories - example}
\onslide<1-7>{
$
\begin{array}{lcl}
p &\equiv& 3<x \\
q &\equiv& x < 0 \\
r &\equiv& x < y \\
s &\equiv& y < 0 \\
\end{array}
$}

\vspace{2mm}
\[
\begin{array}{lclcl}
\onslide<2-7>{
\emptyset &\parallel & \red{p}, q\vee r,s\vee\neg r &\Longrightarrow& (\mbox{\tt Unit Propagate}) \\[2mm]
}
\onslide<3-7>{
p &\parallel & p, q\vee r,s\vee\neg r &\Longrightarrow& (\mbox{\tt Decide}) \\[2mm]
}
\onslide<4-7>{
p,q^d &\parallel & p, q\vee r,s\vee\neg r &\Longrightarrow& (\mbox{\tt Decide}) \\[2mm]
}
\onslide<5-7>{
p,q^d,r^d &\parallel & p, q\vee r,\red{s\vee\neg r} &\Longrightarrow& (\mbox{\tt Unit Propagate}) \\[2mm]
}
\onslide<6-7>{
p,q^d,r^d,s &\parallel & p, q\vee r,s\vee\neg r &\Longrightarrow& (\mbox{\tt T-Learn}) \\[2mm]
&&&&\red{\emptyset\models_\cT \neg (3<x)\vee \neg(x<0)} \\[2mm]
}
\onslide<7-7>{
p,q^d,r^d,s &\parallel & p, q\vee r,s\vee\neg r,\neg p\vee\neg q &\Longrightarrow& (\mbox{\tt Restart}) \\[2mm]
\emptyset &\parallel & p, q\vee r,s\vee\neg r,\neg p\vee\neg q &&
}
\end{array}
\]
\end{frame}

%---------------------------------------------------------------------- SLIDE -
\begin{frame}
\frametitle{Abstract DPLL modulo theories - example}
\onslide<1-6>{
$
\begin{array}{lcl}
p &\equiv& 3<x \\
q &\equiv& x < 0 \\
r &\equiv& x < y \\
s &\equiv& y < 0 \\
\end{array}
$}

\vspace{2mm}
\[
\begin{array}{lclcl}
\onslide<1-6>{
\emptyset &\parallel & \red{p}, q\vee r,s\vee\neg r,\neg p\vee\neg q &\Longrightarrow& (\mbox{\tt Unit Propagate}) \\[2mm]
}
\onslide<2-6>{
p &\parallel & p, q\vee r,s\vee\neg r,\red{\neg p\vee\neg q} &\Longrightarrow& (\mbox{\tt Unit Propagate}) \\[2mm]
}
\onslide<3-6>{
p,\neg q &\parallel & p, \red{q\vee r},s\vee\neg r,\neg p\vee\neg q &\Longrightarrow& (\mbox{\tt Unit Propagate}) \\[2mm]
}
\onslide<4-6>{
p,\neg q,r &\parallel & p, q\vee r,\red{s\vee\neg r},\neg p\vee\neg q &\Longrightarrow& (\mbox{\tt Unit Propagate}) \\[2mm]
}
\onslide<5-6>{
p,\neg q,r,s &\parallel & p, q\vee r,s\vee\neg r,\neg p\vee\neg q &\Longrightarrow& (\mbox{\tt T-Learn}) \\[2mm]
}
\end{array}
\]
\onslide<5-6>{
\[
\red{\emptyset\models_\cT \neg (3<x)\vee \neg(x<y)\vee \neg(y<0)}
\]}
\[
\begin{array}{lclcl}
\onslide<6-6>{
p,\neg q,r,s &\parallel & p, q\vee r,s\vee\neg r,\neg p\vee\neg q,\neg p\vee \neg r\vee \neg s &\Longrightarrow& (\mbox{\tt Restart}) \\[2mm]
\emptyset &\parallel & p, q\vee r,s\vee\neg r,\neg p\vee\neg q,\neg p\vee \neg r\vee \neg s && 
}
\end{array}
\]
\end{frame}

%---------------------------------------------------------------------- SLIDE -
\begin{frame}
\frametitle{Abstract DPLL modulo theories - example}
\onslide<1-5>{
$
\begin{array}{lcl}
p &\equiv& 3<x \\
q &\equiv& x < 0 \\
r &\equiv& x < y \\
s &\equiv& y < 0 \\
\end{array}
$
}

\vspace{2mm}
\[
\begin{array}{lclcl}
\onslide<1-5>{
\emptyset &\parallel & \red{p}, q\vee r,s\vee\neg r,\neg p\vee\neg q,\neg p\vee \neg r\vee \neg s & (\mbox{\tt Unit Propagate}) \\[2mm]
}
\onslide<2-5>{
p &\parallel & p, q\vee r,s\vee\neg r,\red{\neg p\vee\neg q},\neg p\vee \neg r\vee \neg s & (\mbox{\tt Unit Propagate}) \\[2mm]
}
\onslide<3-5>{
p,\neg q &\parallel & p, \red{q\vee r},s\vee\neg r,\neg p\vee\neg q,\neg p\vee \neg r\vee \neg s & (\mbox{\tt Unit Propagate}) \\[2mm]
}
\onslide<4-5>{
p,\neg q,r &\parallel & p, q\vee r,\red{s\vee\neg r},\neg p\vee\neg q,\neg p\vee \neg r\vee \neg s & (\mbox{\tt Unit Propagate}) \\[2mm]
}
\onslide<5-5>{
p,\neg q,r,s &\parallel & p, q\vee r,s\vee\neg r,\neg p\vee\neg q,\neg p\vee \neg r\vee \neg s & (\mbox{\tt Fail}) 
}
\end{array}
\]
\onslide<5-5>{
\[
\red{p,\neg q,r,s \models \neg (\neg p\vee \neg r\vee \neg s)}
\]
}
\end{frame}

%\section{Combination of theories}
%
%\begin{frame}
%\frametitle{SMT solvers}
%\tableofcontents[currentsection]
%\end{frame}
%
%
%
%%---------------------------------------------------------------------- SLIDE -
%\begin{frame}
%\frametitle{Combination of theories}
%\begin{itemize}
%\setlength\itemsep{2mm}
%\item In practice, we need to reason in several theories at once
%\item Example: $1\leq i\leq n \limp a[i] = f(i)$
%\end{itemize}
%
%\vspace{3mm}
%\red{Theory combination:}
%
%\vspace{-3mm}
%\begin{tabular}{ll}
%\quad \quad & \\
%&$\Sigma_1$ and $\Sigma_2$ are disjoint signatures: $\Sigma_1\cap\Sigma_2=\emptyset$ ($=$ is a logical symbol) \\[2mm]
%&$\cT_1$ and $\cT_2$ are theories over $\Sigma_1$ and $\Sigma_2$\\[2mm]
%&$\cT$ = the set of deductively closed sentences that includes $\cT_1\cup \cT_2$
%\end{tabular}
%
%\vspace{3mm}
%\blue{
%Given satisfiability procedures for $\cT_1$ and $\cT_2$, build a satisfiability procedure for $\cT$}
%\end{frame}
%
%%---------------------------------------------------------------------- SLIDE -
%\begin{frame}
%\frametitle{Consistency of the combined theory}
%\vspace{3mm}
%\red{Theory combination:}
%
%\vspace{-3mm}
%\begin{tabular}{ll}
%\quad \quad & \\
%&$\Sigma_1$ and $\Sigma_2$ are disjoint signatures: $\Sigma_1\cap\Sigma_2=\emptyset$ \\[2mm]
%&$\cT_1$ and $\cT_2$ are theories over $\Sigma_1$ and $\Sigma_2$\\[2mm]
%&$\cT$ = the set of deductively closed sentences that includes $\cT_1\cup \cT_2$
%\end{tabular}
%
%\vspace{3mm}
%\blue{
%Is the theory $\cT$ consistent (i.e., $\mathit{false}\not\in\cT$) ?}
%
%\end{frame}
%
%%---------------------------------------------------------------------- SLIDE -
%\begin{frame}
%\frametitle{Consistency of the combined theory}
%\begin{theorem}
%Let $\cT_1$ and $\cT_2$ be two consistent theories over disjoint signatures $\Sigma_1$ and $\Sigma_2$.
%Their union is consistent iff there is a cardinal $k$ such that both $\cT_1$ and $\cT_2$ have a model of cardinality $k$.
%\end{theorem}
%\begin{proof}
%($\Rightarrow$) Since $\cT$ is consistent, there exists a model $M$ of $\cT$. 
%
%By construction of $\cT$, the projection of $M$ on $\Sigma_i$ is a model of $\cT_i$, $i=1,2$.
%
%Both projections have the same cardinality.
%
%($\Leftarrow$) Consequence of Craig-Robinson Theorem $\ldots$
%\end{proof}
%
%\end{frame}
%
%%Let A and B be sentences such that A j= B. Let L1 be the language consisting
%%of the symbols which occur in A and let L2 be the language consisting of the symbols which
%%occur in B. Then there is a sentence  in the language L0 = L1 \ L2 such that A j=  j= B.
%
%%---------------------------------------------------------------------- SLIDE -
%\begin{frame}
%\frametitle{Nelson-Oppen Combination Procedure}
%\setbeamercolor{prog}{fg=black,bg=white}
%\begin{center}
%\begin{beamerboxesrounded}[ht=5cm,center,wd=12cm]{}
%\red{Input:} $\varphi$ a conjunction of literals over $\Sigma_1\cup\Sigma_2$
%
%\vspace{2mm}
%\begin{description}
%\item[Purification:] Preserving satisfiability, transform $\varphi$ into $\varphi_1\land\varphi_2$
%such that $\varphi_i$ uses only symbols in $\Sigma_i$, $i=1,2$ 
%\item[Interaction:] Guess an equivalence relation over $\cV(\varphi_1)\cap \cV(\varphi_2)$. Express it
%as a conjunction of literals $\psi$
%\item[Component procedures:] Use the individual procedures to decide the satisfiability of $\varphi_i\land\psi$, $i=1,2$.
%\end{description}
%
%\vspace{2mm}
%\red{Output:} If both are satisfiable then return satisfiable. Otherwise, unsatisfiable.
%\end{beamerboxesrounded}
%\end{center}
%\end{frame}
%
%%---------------------------------------------------------------------- SLIDE -
%\begin{frame}
%\frametitle{Nelson-Oppen Combination Procedure - example}
%\onslide<1-11>{
%\[
%\onslide<1-2>{{\color<2-2>{red}{x+2 = y}} \land} f(
%\onslide*<1-8>{
%{\color<8-8>{red}
%read(write(a,x,
%\onslide*<1-4>{{\color<4-4>{red} 3}}
%\onslide*<5-11>{u_1}
%),
%\onslide*<1-6>{{\color<6-6>{red} y-2}}
%\onslide*<7-11>{u_2}
%)}}
%\onslide*<9-11>{u3}
%)
%\neq f(
%\onslide*<1-10>{{\color<10-10>{red} y-x+1}}
%\onslide*<11-11>{u_4}
%)
%\]}
%
%\vspace{4mm}
%\begin{center}
%\begin{tabular}{c|c|c}
%Linear integer\quad\quad\quad & Arrays\quad\quad\quad & EUF\quad\quad\quad \\
%Arithmetic && \\ \hline 
%\onslide<3->{$x+2=y$}& \onslide<9->{$u_3=read(write(a,x,u_1),u_2)$} & \onslide<12->{$f(u_3)\neq f(u_4)$}\\
%\onslide<5->{$u_1=3$}& \onslide<15->{$u_2=x$} & \onslide<15->{$u_2=x$}\\
%\onslide<7->{$u_2=y-2$}& \onslide<15->{$u_4=u_1$} & \onslide<15->{$u_4=u_1$}\\ 
%\onslide<11->{$u_4=y-x+1$} & \onslide<16->{{\color<16-17>{red} $u_3=u_1$}} & \onslide<17->{$u_3=u_1$}\\
%\onslide<13->{{\color<13-15>{red} $u_2=x$}} && \onslide<18->{\red{UNSAT}}\\
%\onslide<14->{{\color<14-15>{red} $u_4=u_1$}} \\
%\onslide<17->{$u_3=u_1$} 
%\end{tabular}
%\end{center}
%\end{frame}
%
%%---------------------------------------------------------------------- SLIDE -
%\begin{frame}
%\frametitle{Nelson-Oppen \red{Deterministic} Combination Procedure}
%\begin{itemize}
%\onslide<1->{
%\setlength\itemsep{2mm}
%\item Instead of \red{guessing}, we can \red{deduce} the shared equalities
%\item Works for \red{convex} theories
%	\vspace{2mm}
%	\begin{itemize}
%	\setlength\itemsep{2mm}
%		\item a theory $\cT$ is convex iff for all finite sets of literals $\Gamma$ and for all non-empty disjunctions 
%		$\bigvee_{i} x_i=y_i$,
%		\[
%		\Gamma\models_\cT \bigvee_{i} x_i=y_i \mbox{ iff }\Gamma\models_\cT x_i=y_i,\mbox{ for some $i$}
%		\]
%		\item linear rational arithmetic is convex
%		\item linear integer arithmetic is not convex: $y=1,z=2,1\leq x\leq 2\models x=y\vee x=z$
%	\end{itemize}
%\item For non-convex theories deduce \red{disjunctions} of equalities $\leadsto$ not efficient
%}
%\onslide<2->{
%\item \red{Model-based theory combination}
%	\vspace{2mm}
%	\begin{itemize}
%	\setlength\itemsep{2mm}
%		\item use a candidate model for one of the theories $\cT_i$, and propagate all equalities implied by this models, hoping that the other theories will agree
%		\item if not, backtrack to fix the model
%	\end{itemize}
%}
%\end{itemize}
%\end{frame}
%
%%---------------------------------------------------------------------- SLIDE -
%\begin{frame}
%\frametitle{Nelson-Oppen Combination Procedure}
%\setbeamercolor{prog}{fg=black,bg=white}
%\begin{beamerboxesrounded}[ht=5cm,center,wd=12cm]{}
%\red{Input:} $\varphi$ a conjunction of literals over $\Sigma_1\cup\Sigma_2$
%
%\vspace{2mm}
%\begin{description}
%\item[Purification:] Preserving satisfiability, transform $\varphi$ into $\varphi_1\land\varphi_2$
%such that $\varphi_i$ uses only symbols in $\Sigma_i$, $i=1,2$ 
%\item[Interaction:] Guess an equivalence relation over $\cV(\varphi_1)\cap \cV(\varphi_2)$. Express it
%as a conjunction of literals $\psi$
%\item[Component procedures:] Use the individual procedures to decide the satisfiability of $\varphi_i\land\psi$, $i=1,2$.
%\end{description}
%
%\vspace{2mm}
%\red{Output:} If both are satisfiable then return satisfiable. Otherwise, unsatisfiable.
%\end{beamerboxesrounded}
%
%\begin{itemize}
%\setlength\itemsep{2mm}
%\item A theory is \red{stably infinite} if every satisfiable quantifier-free formula is satisfiable in an infinite model
%\end{itemize}
%
%\begin{theorem}
%Let $\cT_1$ and $\cT_2$ be two consistent stably-infinite theories over disjoint signatures $\Sigma_1$ and $\Sigma_2$.
%A conjunction of literals $\varphi$ over $\Sigma_1\cup\Sigma_2$ is satisfiable iff there exist some $\psi$ such that $\varphi_1\land\psi$ and $\varphi_2\land\psi$ are satisfiable.
%\end{theorem}
%\end{frame}
%
%%---------------------------------------------------------------------- SLIDE -
%\begin{frame}
%\frametitle{Why stably-infinite ?}
%\begin{itemize}
%\item Theories that admit only finite model pose problems
%\end{itemize}
%\begin{example}
%\begin{itemize}
%	\item $\cT_1$=a theory that admits only models of cardinality \red{at most 2}
%	\item $\cT_2$=a theory that admits models of any cardinality 
%	\item the combined theory is consistent
%	\item let \blue{$\varphi:\ f(x)\neq f(y)\land g(x)\neq g(z)\land g(y)\neq g(z)$}
%	\item $\varphi$ is split into \blue{$\varphi_1:\ f(x)\neq f(y)$} and \blue{$\varphi_2:\ g(x)\neq g(z)\land g(y)\neq g(z)$}
%	\item possible equivalence classes: $\psi_1:\ x=y$, \blue{$\psi_2:\ x\neq y$}
%	\item $\varphi_1\land\psi_2$ and $\varphi_2\land\psi_2$ are satisfiable $\leadsto$ \red{conclude $\varphi$ 
%satisfiable}
%	\item but, $\varphi$ is \red{unsatisfiable} because $\cT$ as $\cT_1$ admits only models of cardinality at most 2.
%\end{itemize}
%\end{example}
%\end{frame}

%---------------------------------------------------------------------- SLIDE -
\begin{frame}
\frametitle{ }
\begin{center}
\red{\Huge{The end}}
\end{center}
\end{frame}


%%---------------------------------------------------------------------- SLIDE -
%\begin{frame}
%\frametitle{ }
%\begin{itemize}
%\item 
%\end{itemize}
%\end{frame}


\end{document}

